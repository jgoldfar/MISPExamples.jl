% !TEX root = main.tex
Here we develop a model problem to test implementations of the developed numerical method.
One such problem can be found in the text by Tikhonov \& Samarskii~\cite[Ch.\ II, App.\ IV, pp.\ 283--288]{tikhonov63}.
The model problem originates in the following: find \(\big(u, \xi{}\big)\) satisfying
\begin{gather}
  \D{u_1}{t} - k_1 \D{^2 u_1}{x^2} = 0,~0<x<\xi(t),~t>0\label{eq:tikhonov-samarskii-pde-1}
  \\
  \D{u_2}{t} - k_2 \D{^2 u_2}{x^2} = 0,~\xi(t)<x<\infty,~t>0\label{eq:tikhonov-samarskii-pde-2}
  \\
  u_1(0,t) = c_1,~t>0
  \\
  u_2(x,0) = c_2,~x>\xi(0)
  \\
  u_1(x,0) = c_1,~x<\xi(0) % Note: this condition is added since $\xi$ is
  % unknown a-priori.
  \\
  u_1(\xi(t),t) = u_2(\xi(t),t)=0,~t>0
  \\
  k_1 \D{u_1}{x}(\xi(t),t) - k_2 \D{u_2}{x}(\xi(t),t) = \gamma \dD{\xi}{t}(t)\label{eq:tikhonov-samarskii-stefancond}
\end{gather}
where \(k_1, k_2, \gamma >0\) are given, and without loss of generality \(c_1<0\), \(c_2\geq 0\).
Direct calculation verifies that the exact solution to~\eqref{eq:tikhonov-samarskii-pde-1}--\eqref{eq:tikhonov-samarskii-stefancond} is
\begin{gather}
  u_1(x,t)=c_1 + B_1 \mathrm{erf}\left( \frac{x}{\sqrt{4 k_1 t}}\right)
  \\
  u_2(x,t)=A_2 + B_2 \mathrm{erf}\left( \frac{x}{\sqrt{4 k_2 t}}\right)
  \\
  \xi(t) = \alpha \sqrt{t}
  \intertext{where}
  \mathrm{erf}(x)=\frac{2}{\sqrt{\pi}} \int_0^x e^{-z^2} \,dz,
  \\
  B_1 =-\frac{c_1}{\mathrm{erf}\left(\alpha/\sqrt{4 k_1}\right)},\quad
  B_2 = \frac{c_2}{1-\mathrm{erf}\left(\alpha/\sqrt{4 k_2}\right)}
  \\
  A_2 =-\mathrm{erf}\left(\alpha/\sqrt{4 k_2}\right) B_2
  \intertext{and \(\alpha{}\) is a solution of the transcendental equation}
  % %% Directly translated
  %\frac{k_1 c_1 e^{-\frac{\alpha^2}{4 k_1}}}{\sqrt{k_1} \Phi\left(\alpha/\sqrt{4 k_1}\right)}
  %+ \frac{k_2 c_2 e^{-\frac{\alpha^2}{4 k_2}}}{\sqrt{k_2}\left[1-\Phi\left(\alpha/\sqrt{4 k_2}\right) \right]}
  %= -\frac{\gamma \sqrt{\pi}}{2} \alpha
  % %% But some cancellation happens since we chose particular values for
  % a_1/a_2 from reference.
  \frac{\sqrt{k_1} c_1 e^{-\frac{\alpha^2}{4 k_1}}}{ \mathrm{erf}\left(\alpha/\sqrt{4 k_1}\right)}
  + \frac{\sqrt{k_2} c_2 e^{-\frac{\alpha^2}{4 k_2}}}{\left[1-\mathrm{erf}\left(\alpha/\sqrt{4 k_2}\right) \right]}
  = -\frac{\gamma \sqrt{\pi}}{2} \alpha
\end{gather}
The equation above has a unique solution \(\alpha>0\).
To transform this problem to the form found in~\cite{abdulla16a}, the functions
\(\alpha(u)\), \(k(u)\), etc.\ should be found, along with the phase transition
temperatures \(u^j\) and other problem data.
Choose the time domain \(0<t<T=: 1\), so the maximal extent of the first phase is \(\xi(T) = \alpha\); we choose \(\l = 2\alpha{}\).
The problem above is a two phase problem with phase transition temperature \(u^1
= 0\), and evidently, we may choose \(\alpha(u)\equiv 1\).
Since \(u=u_1(x,t)\) for \(u<u^1=0\) solves~\eqref{eq:tikhonov-samarskii-pde-1}
and \(u=u_2(x,t)\) for \(u>u^1=0\) solves~\eqref{eq:tikhonov-samarskii-pde-2},
we choose
\[
  k(u) =
  \begin{cases}
    k_1,~&u<0
    \\
    k_2,~&u>0
  \end{cases}
\]
%
Evidently,
\[
  \phi(x) =
  \begin{cases}
    c_1,~&x<\xi(0)
    \\
    c_2,~&x>\xi(0)
  \end{cases}
\]
%
By direct calculation,
\[
  k_1 \D{u_1}{x} \Big\vert_{x=0}
  = g(t) \equiv
  \frac{B_{1}}{\sqrt{\pi} \sqrt{k_{1}} \sqrt{t}} 
%chktex 27 ok (external)
\]
Similarly,
\[
  k_2 \D{u_2}{x} \Big \vert_{x=\l}
  = p(t) \equiv
  \frac{B_{2} e^{- \frac{\alpha^2}{k_{2} t} }}{\sqrt{\pi} \sqrt{k_{2}} \sqrt{t}} 
%chktex 27 ok (external)
\]
Note that while \(p(t)\) is bounded, \(g(t)\) is not.
%
We have
\[
  u_2(\l, t)
  = \nu(t)
  \equiv B_{2} \mathrm{erf}\left(\frac{\alpha}{\sqrt{k_{2}} \sqrt{t}}\right)+A_{2} 
%chktex 27 ok (external)
\]
Note that \(\nu(t)\geq 0\).

We now define the mapping \(F(u)\) as in~\cite{abdulla16a}:
\begin{equation}
  F(u)
  = \int_{u_1}^u k(y) \,dy
  = \int_0^u k(y) \,dy
  = \begin{cases}
    k_1 u,~&u<0
    \\
    k_2 u,~&u>0
    \end{cases}
\end{equation}
The inverse mapping is
\begin{equation}
  F^{-1}(u) =
  \begin{cases}
    \frac{u}{k_1},~&u<0
    \\
    \frac{u}{k_2},~&u>0
  \end{cases}
\end{equation}
In particular we have
\[
  \Gamma(t)
  = \int_0^{\nu(t)} k(u) \,du
  = \begin{cases}
    k_1 \nu(t),~&\nu(t)<0
    \\
    k_2 \nu(t),~&\nu(t)>0
  \end{cases}
  = k_2 \nu(t)
\]
since \(\nu \geq 0\).
and
\[
  \Phi(x) =
  \begin{cases}
    k_1 c_1,~&x<\xi(0)
    \\
    k_2 c_2,~&x>\xi(0)
  \end{cases}
\]
Having \(F\), \(\alpha{}\), and \(k\), we can define
\begin{gather}
  \beta(v)
  = \frac{\alpha(F^{-1}(v))}{k(F^{-1}(v))}
  = \frac{1}{k(F^{-1}(v))}\nonumber
  %\\
  = \begin{cases}
    \frac{1}{k(v/k_1)},~&v<0
    \\
    \frac{1}{k(v/k_2)},~&v>0
  \end{cases}\nonumber
  %\\
  =\begin{cases}
    \frac{1}{k_1},~&v<0
    \\
    \frac{1}{k_2},~&v>0
  \end{cases}
\end{gather}
since \(k_1, k_2 > 0\).
Now we introduce a function \(b(v)\) such that \(b'(v) = \beta(v)\);
we additionally require that \([b(v)]_{v=v^1} = [b(v)]_{v=0} = \gamma{}\), and so
\begin{equation}
  b(v) =
  \begin{cases}
    \gamma + \frac{v}{k_2},~&v>0
    \\
    \frac{v}{k_1},~&v<0
  \end{cases}
\end{equation}
% For any \(\epsilon>0\) define
% \begin{equation}
%   b_{\epsilon}(v) = \int_{-\epsilon}^{\epsilon} f(v-z) \eta_{\epsilon}(z) \,dz
% \end{equation}
% where \(\eta_{\epsilon}\) is a mollifier supported in \((-\epsilon,\epsilon)\).
%
To avoid issues with the lack of boundedness of the given data, the right approach is to shift the problem in time.
The requirement that the initial data be constant is not essential.

To summarize, the model problem of Tikhonov \& Samarskii~\cite[Ch.\ II, App.\
IV, pp.\ 283--288]{tikhonov63} is translated to the framework of Abdulla \&
Poggi~\cite{abdulla16a} as follows: fix any \(t_0, k_1, k_2>0\), \(c_1<0\),
\(c_2 \geq 0\).
Find the corresponding \(\alpha>0\) as defined above, and set \(\xi_0 = \alpha
\sqrt{t_0}\) and \(\l=2\alpha\).
Find \(v\), \(\xi{}\), and one of \(p\), \(g\), or \(f\) satisfying
\begin{gather}
  \D{b(v)}{t} - v_{xx}
  = f~\text{in}~
  D=\bk{(x,t):~0<x<\l:=2\alpha,~t_0 < t < T:=1}\label{eq:tikhonov-samarskii-pde}
  \\
  v(x,0)
  = \Phi(x)
  :=\begin{cases}
    k_1 u_1(x,t_0),~x < \xi_0
    \\
    k_2 u_2(x,t_0),~x > \xi_0
  \end{cases}
  \\
  v_x(0,t) = g(t),\quad
  v_x(\l, t) = p(t),\quad 0<t<T\label{eq:tikhonov-samarskii-neumann-conditions}
  \\
  v(\l, t) = \Gamma(t) := k_2 \nu(t) = k_2 \left(
    B_{2} \mathrm{erf}\left(\frac{\alpha}{\sqrt{k_{2}} \sqrt{t}}\right)+A_{2} 
%chktex 27 ok (external)
  \right)\label{eq:tikhonov-samarskii-extra-measurement}
\end{gather}

The variational formulation of the
problem~\eqref{eq:tikhonov-samarskii-pde}--\eqref{eq:tikhonov-samarskii-extra-measurement}
is as follows: consider the minimization of the functional
\[
  \fJ{}(c) = \int_0^T \norm{v(\l, t; c) - \Gamma(t)}^2 \,dt
\]
where \(c\) belongs to the control set
\[
  V_R = \bk{c=(\controlVars{}) \in H,~\Norm{c}_{H} \leq R}
\]
and the state vector \(v=v(x,t;c)\) is a weak solution
of the Neumann problem~\eqref{eq:tikhonov-samarskii-pde}--\eqref{eq:tikhonov-samarskii-neumann-conditions};
that is, \(v\in\stateVectorSpace{}\) satisfies
\begin{gather*}
  \iint_D\Big[
  -b(x,t,v(x,t))\psi_t
  +v_x\psi_x
  -f\psi
  \Big]\,dx\,dt
  - \int_0^{\ell}b(x,0,\Phi(x))\psi(x,0)\,dx \nonumber
  \\
  - \int_0^T p(t)\psi(\ell,t)\,dt
  +\int_0^T g(t)\psi(0,t)\,dt
  = 0,
  \qquad \forall \psi\in W_2^{1,1}(D),
  \quad \psi(x,T)=0.
\end{gather*}
Now that all of the ``analytic'' calculations are completed, we move to the
development of the numerical method.
Let \(m\), \(n\), \(T\), \(\l{}\) be given, and define the space and time grids
\(\omega_\tau{}\), \(\omega_h\) as in~\cite{abdulla16a}.
In particular, define \(\tau = T/n\), \(h=\l/m\), \(t_k = k \tau{} + t_0\), and \(x_i
= i h\).
Given this grid, define
\begin{gather*}
  \Gamma_k
  = \frac{1}{\tau} \int_{t_{k-1}}^{t_k} \Gamma(t) \,dt
  = \frac{1}{\tau} \int_{t_{k-1}}^{t_k} k_2 \nu(t) \,dt
  \\
  = k_2 A_2 + \frac{k_2 B_2}{\tau} \int_{t_{k-1}}^{t_k} \mathrm{erf}\left(
    \frac{1}{\sqrt{4 k_2 t}}
  \right) \,dt
\end{gather*}
and
\begin{gather*}
  \Phi_i
  = \frac{1}{h} \int_{x_i}^{x_{i+1}} \Phi(x) \,dx
\end{gather*}
One might choose to take \(g_k\), \(p_k\), or \(f_{i}(k)\) as the corresponding
Steklov average of the analytic values, but in general the numerical method is
to consider the minimization of the functional
\begin{gather*}
  I_n([c]_n) = \sum_{k=1}^n \tau \left( v_m(k) - \Gamma_k \right)^2
  \intertext{where the control \([c]_n\) belongs to the control set}
  \discreteControlSpace{} = \Big{\{}
  [c]_n = \big( \discreteControlVars{}\big),~
  \Norm{[c]_n}_{\discreteControlSet{}} \leq R
  \Big{\}}
  \intertext{and the state vector satisfies \(v_i(0) = \Phi_i\) and}
  \sum_{k=1}^n \tau \sum_{i=0}^{m-1} h\left[
    \big(b_n(v_i(k))\big)_{\bar{t}}\eta_i(k)
    + v_{ix}(k)\eta_{ix}(k)
    - f_i(k) \eta_i(k)
  \right]
  - p_k \eta_m(k)
  + g_k \eta_0(k)
  = 0
\end{gather*}
for all \(\bk{\eta_i(k)} \subset \R^{nm}\).

Tikhonov \& Samarskii remark that if \(c_{2}\) is equal to the temperature of fusion (\(c_{2}=0\)) then the coefficients take the simpler form
\begin{gather*}
A_{2} = B_{2} = 0,\quad
% \\
A_{1} = c_{1},\quad
% \\
B_{1} = -\frac{1}{\mathrm{erf}\left( \frac{\alpha}{\sqrt{4 k_{2}}}\right)}
\intertext{and \(\alpha{}\) is a solution of the transcendental equation}
\frac{\sqrt{k_1} c_1 e^{-\frac{\alpha^2}{4 k_1}}}{ \mathrm{erf}\left(\alpha/\sqrt{4 k_1}\right)}
  = -\frac{\gamma \sqrt{\pi}}{2} \alpha
\end{gather*}