%!TEX root = main.tex
\newcommand\field[1]{\mathbb{#1}}
\newcommand\R{\field{R}}
\newcommand\ZZ{\field{Z}}
\newcommand\bk[1]{\left\{ #1 \right\}} % chktex 21 ok (left/right brace)
\newcommand\ip[2]{\left\langle{} #1, #2 \right\rangle}
\newcommand\D[2]{\frac{\partial{} #1}{\partial{} #2}}
\newcommand\dD[2]{\frac{d #1}{d #2}}
\newcommand{\nc}{\newcommand}
\nc{\ep}{\varepsilon}
\nc{\n}[1]{\mathcal{#1}}
\nc{\m}[1]{\mathcal{#1}}
\newcommand{\Norm}[1]{\Vert{}#1\Vert}
\newcommand{\norm}[1]{\vert{}#1\vert}

% for indicator function
\newcommand{\Ind}{\mathbf{1}}
\DeclareMathOperator*{\esssup}{ess~sup}
% \DeclareMathOperator{\sign}{sign}
%
% \makeatletter
% \def\ifdraft{\ifdim\overfullrule>\z@
%   \expandafter\@firstoftwo\else\expandafter\@secondoftwo\fi} % chktex 41 ok % chktex 21 ok (needed to define ifdraft flag)
% \makeatother
